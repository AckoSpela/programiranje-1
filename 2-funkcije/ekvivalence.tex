\documentclass[11pt,a4paper]{article}
\usepackage{a4wide}
\usepackage{amsmath}
\usepackage[slovene]{babel}
\usepackage[utf8]{inputenc}
\usepackage{titlesec}

\begin{document}

\begin{center}
  \huge \textbf{Dokazovanje ekvivalenc med programi}
\end{center}
\newcommand{\inline}[1]{\mbox{\texttt{#1}}}
\bigskip

\titleformat{\section}
{\normalfont\Large\bfseries}
{Trd.~\thesection:}{0.3em}{}

\noindent
Dostikrat si želimo dokazati, da nek program deluje tako, kot želimo.
Običajno je to precej zahtevna naloga, saj so programski jeziki nepredvidljivi.
Ker pa je Haskell čist jezik, so v njem izrazi enakovredni svojim definicijam, zato lahko ene zamenjamo z drugimi in obratno.
Da bomo imeli kaj, s čimer bomo lahko delali, si najprej definirajmo sledeče tri funkcije:

\begin{align}
  \inline{[] ++ ys} &= \inline{ys} \label{app1} \\
  \inline{(x:xs) ++ ys} &= \inline{x:(xs ++ ys)} \label{app2} \\[1em]
  \inline{obrni []} &= \inline{[]} \label{obrni1} \\
  \inline{obrni (x:xs)} &= \inline{obrni xs ++ [x]} \label{obrni2} \\[1em]
  \inline{dolzina []} &= \inline{0} \label{dolzina1} \\
  \inline{dolzina (x:xs)} &= \inline{1 + dolzina xs} \label{dolzina2}
\end{align}

\section{$\inline{obrni [x]} = \inline{[x]}$}

To enostavno trditev dokažemo zgolj z odvijanjem definicij.

\begin{align*}
  \inline{obrni [x]}
  &= \inline{obrni (x:[])}
    &&\text{ker je $\inline{[x]}$ okrajšava za $\inline{x:[]}$} \\
  &= \inline{[] ++ [x]}
    &&\text{po \eqref{obrni1}} \\
  &= \inline{[x]}
    &&\text{po \eqref{app1}}
\end{align*}

\section{$\inline{dolzina (xs ++ ys)} = \inline{dolzina xs + dolzina ys}$}

Tukaj enostavno odvijanje definicij ne bo pomagalo, saj izraz $\inline{dolzina (xs ++ ys)}$ ne ustreza ne levi ne desni strani nobene od zgoraj naštetih definicij.
Namesto tega uporabimo načelo indukcije za sezname:
\[
  P(\inline{[]}) \land (\forall \inline{x}, \inline{xs}. P(\inline{xs}) \Rightarrow P(\inline{x:xs})) \implies \forall \inline{ys}. P(\inline{ys})
\]
Torej, lastnost $P$ velja za vse sezname $\inline{ys}$, kadar (1) velja za prazen seznam $\inline{[]}$ in (2) velja za sestavljen seznam $\inline{x:xs}$ ob predpostavki, da velja za $\inline{xs}$.
Načelo indukcije je podobno načelu indukcije za naravna števila:
\[
  P(0) \land (\forall m. P(m) \Rightarrow P(m^{+})) \implies \forall n. P(n)
\]
Torej, lastnost $P$ velja za vsa naravna števila $n$, kadar (1) velja za $0$ in (2) velja za naslednika $m^{+}$ ob predpostavki, da velja za $m$.

Izjavo dokažemo z indukcijo na levi seznam. Indukcija poteka v dveh korakih:

\subsection*{Osnovni korak}

V osnovnem koraku pokažemo, da enakost velja, kadar je levi seznam prazen, torej oblike $\inline{[]}$:

\begin{align*}
  \inline{dolzina ([] ++ ys)}
  &= \inline{dolzina ys}
    &&\text{po \eqref{app1}} \\
  &= \inline{0 + dolzina ys}
    &&\text{po Peanovih aksiomih} \\
  &= \inline{dolzina [] + dolzina ys}
    &&\text{po \eqref{dolzina1}}
\end{align*}

\subsection*{Indukcijski korak}

V indukcijskem koraku pokažemo, da enakost velja za sestavljeni seznam $\inline{x:xs}$ ob predpostavki, da enakost velja za seznam $\inline{xs}$:

\begin{align*}
  \inline{dolzina ((x:xs) ++ ys)}
  &= \inline{dolzina (x:(xs ++ ys))}
    &&\text{po \eqref{app2}} \\
  &= \inline{1 + dolzina (xs ++ ys)}
    &&\text{po \eqref{dolzina2}} \\
  &= \inline{1 + (dolzina xs + dolzina ys)}
    &&\text{po indukcijski prepostavki} \\
  &= \inline{(1 + dolzina xs) + dolzina ys)}
    &&\text{po Peanovih aksiomih} \\
  &= \inline{dolzina (x:xs) + dolzina ys}
    &&\text{po \eqref{dolzina2}}
\end{align*}


\section{$\inline{xs ++ []} = \inline{xs}$}
\label{sec:desna-enota}

Po \eqref{app1} vemo, da je $\inline{[] ++ xs} = \inline{xs}$, torej je $\inline{[]}$ leva enota za $\inline{++}$. To, da je $\inline{[]}$ tudi desna enota, pa ni samoumevno --- za dokaz uporabimo indukcijo.

\subsection*{Osnovni korak}

Po \eqref{app1} velja $\inline{[] ++ []} = \inline{[]}$, kar dokaže osnovni korak.

\subsection*{Indukcijski korak}

Prepostavimo, da velja $\inline{xs ++ []} = \inline{xs}$. Tedaj velja

\begin{align*}
  \inline{(x:xs) ++ []}
  &= \inline{x:(xs ++ [])}
    &&\text{po \eqref{app2}} \\
  &= \inline{x:xs}
    &&\text{po indukcijski prepostavki},
\end{align*}

kar zaključi tudi indukcijski korak.


\section{$\inline{xs ++ (ys ++ zs)} = \inline{(xs ++ ys) ++ zs}$}
\label{sec:asociativnost}

Operacija stikanja seznamov $\inline{++}$ je tudi asociativna, kar dokažemo z indukcijo na $\inline{xs}$. Če zapišete dokaz asociativnosti seštevanja, lahko vidite, da poteka podobno, le da se namesto $\inline{[]}$ pojavlja $0$, namesto $\inline{x:xs}$ pa naslednik $n^{+}$.

\subsection*{Osnovni korak}

\begin{align*}
  \inline{[] ++ (ys ++ zs)}
  &= \inline{ys ++ zs}
    &&\text{po \eqref{app1}} \\
  &= \inline{([] ++ ys) ++ zs}
    &&\text{po \eqref{app1}}
\end{align*}

\subsection*{Indukcijski korak}

Prepostavimo, da velja $\inline{xs ++ (ys ++ zs)} = \inline{(xs ++ ys) ++ zs}$. Tedaj velja

\begin{align*}
  \inline{(x:xs) ++ (ys ++ zs)}
  &= \inline{x:(xs ++ (ys ++ zs))}
    &&\text{po \eqref{app2}} \\
  &= \inline{x:((xs ++ ys) ++ zs)}
    &&\text{po indukcijski predpostavki} \\
  &= \inline{(x:(xs ++ ys)) ++ zs}
    &&\text{po \eqref{app2}} \\
  &= \inline{((x:xs) ++ ys) ++ zs}
    &&\text{po \eqref{app2}}
\end{align*}


\section{$\inline{obrni (xs ++ ys)} = \inline{obrni ys ++ obrni xs}$}

\subsection*{Osnovni korak}

\begin{align*}
  \inline{obrni ([] ++ ys)}
  &= \inline{obrni ys}
    &&\text{po \eqref{app1}} \\
  &= \inline{obrni ys ++ []}
    &&\text{po trditvi~\ref{sec:desna-enota}} \\
  &= \inline{obrni ys ++ obrni []}
    &&\text{po \eqref{obrni1}}
\end{align*}

\subsection*{Indukcijski korak}

\begin{align*}
  \inline{obrni ((x:xs) ++ ys)}
  &= \inline{obrni (x:(xs ++ ys))}
    &&\text{po \eqref{app2}} \\
  &= \inline{obrni (xs ++ ys) ++ [x]}
    &&\text{po \eqref{obrni2}} \\
  &= \inline{(obrni ys ++ obrni xs) ++ [x]}
    &&\text{po indukcijski predpostavki} \\
  &= \inline{obrni ys ++ (obrni xs ++ [x])}
    &&\text{po trditvi~\ref{sec:asociativnost}} \\
  &= \inline{obrni ys ++ obrni (x:xs)}
    &&\text{po \eqref{obrni2}}
\end{align*}


\section{$\inline{obrni (obrni xs)} = \inline{xs}$}

Na kvizu.

\section{$\inline{dolzina (obrni xs)} = \inline{dolzina xs}$}

Domača naloga.

\end{document}
