\documentclass[arhiv]{../izpit}
\usepackage{fouriernc}
\usepackage{xcolor}
\usepackage{tikz}

\begin{document}

\izpit{Programiranje I: 2.\ izpit}{11.\ julij 2017}{
  Čas reševanja je 150 minut.
  \textbf{Do not forget to equip the Haskell features with an appropriate signature.}
  Great success!
}

%%%%%%%%%%%%%%%%%%%%%%%%%%%%%%%%%%%%%%%%%%%%%%%%%%%%%%%%%%%%%%%%%%%%%%
\naloga[Študenti, 20 točk]

Students with Programming 1 can be presented with
Type \texttt{Student}, which we define here:
\begin{verbatim}
  type Student = (Ime, Priimek, VpisnaStevilka, Rezultati)
  type Ime = String
  type Priimek = String
  type VpisnaStevilka = Int
  data Rezultati = Neudelezen | Rezultat (Int, Int, Int, Int) deriving (Show)
\end{verbatim}

Let's save you another year with the next three students:
\begin{verbatim}
  a = ("Ana", "Bertoncelj", 1, Rezultat (10, 20, 15, 2)) :: Student
  b = ("Bine", "Cencelj", 2, Rezultat (5, 13, 20, 17)) :: Student
  c = ("Cilka", "Drnovsek", 3, Neudelezen) :: Student
  
  type Letnik = [Student]
  l = [a, b, c] :: Letnik
\end{verbatim}

\podnaloga
  Sestavite funkcijo \texttt{vsotaTock}, Which calculates the sum of students' points.
  \begin{verbatim}
  ghci> vsotaTock a
  Just 47
  ghci> vsotaTock c
  Nothing
  \end{verbatim}

\podnaloga
  Sestavite funkcijo \texttt{najboljsi}, Who returns the student to the year that has achieved the best result.
  \begin{verbatim}
  ghci> najboljsi l
  ("Bine","Cencelj",2,Rezultat (5,13,20,17))
  \end{verbatim}


%%%%%%%%%%%%%%%%%%%%%%%%%%%%%%%%%%%%%%%%%%%%%%%%%%%%%%%%%%%%%%%%%%%%%%
\naloga[Graf, 20 točk]

The data type \texttt{Graph} is represented as follows:

  \begin{verbatim}
  type Graf = (Vozlisca, Povezave)
  type Vozlisca = [Int]
  type Povezave = [(Int, Int)]

  prazniGraf = ([], []) :: Graf
  \end{verbatim}

Let's save another example of a graph:
  \begin{verbatim}
  v = [1,2,3,4,5] :: Sailor
  p = [(1,2), (1,3), (2,4), (3,5), (4,5)] :: Links
  g = (v, p) :: Graf
  \end{verbatim}

\podnaloga
Compose the \texttt{jeGraf} function to verify that it is input
The graph really is true, so if all the links listed are only between the given nodes.
  \begin{verbatim}
  ghci> jeGraf g
  True
  ghci> jeGraf ([1,2], [(1,3)])
  False
  \end{verbatim}

\podnaloga
We enrich the data type to the type \texttt{PaintedGraf} that has it
Each node is colored. Colors are presented as natural numbers.
  \begin{verbatim}
  type PobarvaniGraf = (Graf, [Barva])
  type Barva = Int
  \end{verbatim}

Compose the \texttt{nicePathValue} function that accepts the graph and
It paints it nicely, that is, it returns a colored graph with a property,
That no two linked nodes are of the same color.
  \begin{verbatim}
  ghci> lepoPobarvaj ([1,2], [])
  (([1,2],[]),[1,1])
  ghci> lepoPobarvaj ([1,2], [(1,2)])
  (([1,2],[(1,2)]),[1,2])
  ghci> lepoPobarvaj g
  (([1,2,3,4,5],[(1,2),(1,3),(2,4),(3,5),(4,5)]),[1,2,2,1,3])
  \end{verbatim}


%%%%%%%%%%%%%%%%%%%%%%%%%%%%%%%%%%%%%%%%%%%%%%%%%%%%%%%%%%%%%%%%%%%%%%
\naloga[Typing, 20 points]
Employees at a large Slovenian bank must enter an account number on each computer after each payment.
For this they use a standard keyboard that has the digits of
0 to 9 are classified in one row. Every employee is the key only with
Cursors and at any moment can do one of two words:
  \begin{enumerate}
    \item Move any of the cursors to the adjacent key, a
      Only one at a time.
    \item Presses the button just below its index finger.
      It can only press one key at a time.
  \end{enumerate}

  In order to carry out each of the above two moves, a staff member needs one
  second. At the beginning, the employee has a left cursor above the 0, the
  right one Above the 9 key. The number is entered from left to right.

Compose the \texttt{typing} function {\em effective}
Calculates the minimum time it takes for an employee to
The computer enters the number.
  \begin{verbatim}
  >>> tipkanje("10")
  4
  >>> tipkanje("107")
  7
  >>> tipkanje("123456789123456789123456789")
  63
  >>> tipkanje("4780")
  13
  >>> tipkanje("01" * 10)
  28
  \end{verbatim}


%%%%%%%%%%%%%%%%%%%%%%%%%%%%%%%%%%%%%%%%%%%%%%%%%%%%%%%%%%%%%%%%%%%%%%
\naloga[The longest interval, 20 points]

Given are $ n $ different integers. Compose the \texttt {farthest}
function As the input receives the list \texttt {sez} of these numbers, and
returns the number set $ (a, b) $ from the input list, which is considered to
contain \texttt{sez} among the numbers $ A $ and $ b $ as many elements, but
all of these elements lie on The numerical interval $ [a, b] $. If there are
more options, let the function return the one at Which pair $ (a, b) $ is the
lexicographic last. The function should work at $ O (n \log n) $.

  Danih je $n$ različnih celih števil. Sestavite funkcijo \texttt{najdaljsi}, ki
  kot vhod sprejme seznam \texttt{sez} teh števil, vrne pa nabor števil $(a,b)$
  iz vhodnega seznama, za kateri velja, da \texttt{sez} vsebuje med številoma
  $a$ in $b$ kar največ elementov, hkrati pa vsi ti elementi ležijo na
  številskem intervalu $[a,b]$. Če je možnosti več, naj funkcija vrne tisto, pri
  kateri je par $(a,b)$ leksikografsko zadnji. Funkcija naj deluje v času $O(n
  \log n)$.

  \begin{verbatim}
  >>> najdaljsi([0,1,2,-1])
  (0,2)
  >>> najdaljsi([9,8,7,6,5,4,3,2,1])
  (9,9)
  >>> najdaljsi([0,1,2,-1,5,7,6,8,-2])
  (-1,8)
  >>> najdaljsi([-10, 0, -1, 0, 1])
  (-1,1)
  \end{verbatim}  

\end{document}

