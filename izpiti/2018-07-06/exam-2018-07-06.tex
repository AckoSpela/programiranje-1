\documentclass[arhiv]{../izpit}
\usepackage{fouriernc}
\usepackage{xcolor}
\usepackage{tikz}
\usepackage{fancyvrb}
\VerbatimFootnotes{}

\begin{document}

\izpit{Programiranje I: 2.\ izpit}{06.\ July 2018}{
  Čas reševanja je 150 minut.
  Veliko uspeha!
}

%%%%%%%%%%%%%%%%%%%%%%%%%%%%%%%%%%%%%%%%%%%%%%%%%%%%%%%%%%%%%%%%%%%%%%
\naloga[]

\podnaloga
Napišite funkcijo
\begin{verbatim}
    uporabi : ('a -> 'b) -> 'a -> 'b
\end{verbatim}
ki uporabi podano funkcijo na podanem argumentu.

\podnaloga
Napišite funkcijo
\begin{verbatim}
    obrnjeno_uporabi : 'a -> ('a -> 'b) -> 'b
\end{verbatim}
ki izvede obrnjeno uporabo. Naprimer, \verb|obrnjeno_uporabi (obrnjeno_uporabi x f) g| je enakovredno $g (f (x))$.

\podnaloga
Napišite funkcijo
\begin{verbatim}
    zacetnih : int -> 'a list -> ('a list) option
\end{verbatim}
tako da \verb|zacetnih n xs| vrne začetnih \verb|n| elementov seznama \verb|xs|, oziroma \verb|None| v primeru, ko ima seznam \verb|xs| manj kot \verb|n| elementov.

Za vse točke naj bo funkcija \emph{repno-rekurzivna}.

\naloga[]
Tip seznamov \verb|'a list| predstavlja prazen nabor ali nabor končno mnogo elementov tipa \verb|'a|. V nadaljevanju bomo obravnavali tip, ki predstavlja zgolj \emph{neprazne} sezname elementov tipa \verb|'a|.

\podnaloga
Definirajte nov tip \verb|'a neprazen_sez| nepraznih seznamov.

\podnaloga
Napišite funkcijo \verb|prvi_el : 'a neprazen_sez -> 'a|, ki vrne prvi element nepraznega seznama.

\podnaloga
Napišite funkcijo \verb|dolzina : 'a neprazen_sez -> int|, ki izračuna dolžino nepraznega seznama.

\podnaloga
Napišite funkcijo \verb|pretvori_v_seznam : 'a neprazen_sez -> 'a list|, ki pretvori neprazen seznam tipa \verb|'a neprazen_sez| v navaden seznam tipa \verb|'a list|.

\podnaloga
Napišite funkcijo \verb|zlozi : ('b -> 'a -> 'b) -> 'b -> 'a neprazen_sez -> 'b|, ki zloži podano funkcijo preko nepraznega seznama. Delovanje funkcije naj bo
podobno delovanju funkcije \verb|List.fold_left|.

\naloga[]
Dr Hannah Habibah je matematičarka, ki se navdušuje nad simetrijo. Po vrnitvi iz izleta v Hajjah, Yemen, je začela iskati simetrije v črtastih črno-belih zapestnicah, ki jih je kupila na potovanju.

Dr Habibah želi razdeliti zaporedje črnih in belih črt na zapestnicah na simetrične dele. Njen cilj je poiskati delitev z najmanjšim številom delov.
Da si delo olajša, se je odločila zaporedje črt predstaviti kot niz ničel in enic.

Obravnava dve vrsti simetrij:
\begin{itemize}
\item del je \emph{p-simetričen} če je palindrom
\item del $D$ dolžine \verb|n| je \emph{vsotno-simetričen} če je vsota prvih
\verb|int(n/2)|\footnote{OCaml: \verb|int_of_float ((float_of_int n) /. 2.)|} števk $D$
  in zadnjih \verb|int(n/2)| števk $D$ enaka
\end{itemize}

\podnaloga
Napišite funkcijo \verb|p_simetricen| ki preveri ali je nek del p-simetričen.
\begin{verbatim}
# Primer:
>>> p_simetricen("01010")
True
\end{verbatim}

\podnaloga
Napišite funkcijo \verb|stevilo_delov|, ki izračuna na koliko najmanj delov moramo razdeliti zaporedje, da so vsi deli p-simetrični.

\begin{verbatim}
# Primer:
>>> stevilo_delov("00101011")
3
\end{verbatim}

\podnaloga
Napišite funkcijo \verb|razdeli|, ki vrne delitev, kjer zaporedje razdelimo na najmanj možno p-simetričnih delov. Če je takšnih delitev več, naj funkcija vrne poljubno pravilno delitev.

\begin{verbatim}
# Primer:
>>> razdeli("00101011")
["0", "01010", "11"]
\end{verbatim}

\podnaloga
Napišite funkcijo \verb|vsotno_simetricen|, ki preveri ali je del vsotno-simetričen.

Namig: za pretvorbo niza števk $b$ v seznam števil uporabite \verb|l = [int(c) for c in b]|\footnote{OCaml: Naložite modul ``Str'' z ukazom \verb| #load "str.cma" ;; |\\ in nato uporabite \verb|l = List.map int_of_string (Str.split (Str.regexp "") b)|}

\begin{verbatim}
# Primer:
>>> vsotno_simetricen("01001000")
True
>>> vsotno_simetricen("1011")
False
\end{verbatim}

\podnaloga
Posplošite funkciji \verb|stevilo_delov| in \verb|razdeli| tako, da dodatno sprejemejo funkcijo, ki preverja ali je nek del simetričen.

\begin{verbatim}
# Primer:
>>> razdeli("00101011", vsotno_simetricen)
["00", "101011"]
\end{verbatim}

\end{document}
