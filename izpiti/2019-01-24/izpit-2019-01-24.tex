\documentclass[arhiv]{../izpit}
\usepackage{fouriernc}
\usepackage{xcolor}
\usepackage{tikz}
\usepackage{fancyvrb}
\VerbatimFootnotes{}

\begin{document}

\izpit{Programiranje I: 1.\ izpit}{24.\ januar 2019}{
  Čas reševanja je 150 minut.
  Veliko uspeha!
}

%%%%%%%%%%%%%%%%%%%%%%%%%%%%%%%%%%%%%%%%%%%%%%%%%%%%%%%%%%%%%%%%%%%%%%%
\naloga[]

\podnaloga
Napišite funkcijo
\begin{verbatim}
    podvoji_vsoto : int -> int -> int,
\end{verbatim}
ki za dani števili izračuna dvakratnik njune vsote.

\podnaloga
Napišite funkcijo
\begin{verbatim}
    povsod_vecji : 'a * 'b * 'c -> 'a * 'b * 'c -> bool,
\end{verbatim}
ki sprejme dve trojici, in preveri, ali ima prva trojica na vseh komponentah večje vrednosti.

\podnaloga
Napišite funkcijo
\begin{verbatim}
    uporabi_ce_lahko : ('a -> 'b) -> 'a option -> 'b option,
\end{verbatim}
ki sprejme funkcijo in element tipa \verb|option| ter v primeru, da element vsebuje vrednost, uporabi funkcijo na vrednosti, sicer vrne \verb|None|.

\podnaloga
Napišite funkcijo
\begin{verbatim}
    pojavi_dvakrat : 'a -> 'a list -> bool,
\end{verbatim}
ki preveri ali se element v seznamu pojavi natanko dvakrat.

\podnaloga
Napišite funkcijo
\begin{verbatim}
    izracunaj_v_tocki : 'a -> ('a -> 'b) list -> 'b list,
\end{verbatim}
ki sprejme točko in seznam funkcij ter vrne seznam rezultatov, ki jih dobimo, ko funkcije izračunamo v točki. Funkcija naj bo repno rekurzivna.

\podnaloga
Napišite funkcijo
\begin{verbatim}
    eksponent : int -> int -> int,
\end{verbatim}
kjer \verb|eksponent x p| izračuna vrednost $x^p$. Funkcija naj bo repno rekurzivna, kar lahko preverite z izračunom števila $1^{1.000.000}$.

%%%%%%%%%%%%%%%%%%%%%%%%%%%%%%%%%%%%%%%%%%%%%%%%%%%%%%%%%%%%%%%%%%%%%%%
\naloga
Podatkovno strukturo dvojiških iskalnih dreves pogosto uporabljamo za predstavitev množic. V tej nalogi bomo drevesa uporabljali za predstavitev multi-množic, kjer se lahko isti element pojavi večkrat, npr. $\{ 1, 1, 2 \}$ sedaj ni več enaka kot $\{ 1, 2 \}$. Da bi se izognili podvajanju vozlišč, sedaj vsako vozlišče dodatno nosi še števec za pojavitve elementa, ohranjamo pa strukturo dvojiških iskalnih dreves. Na primer multi-množico $\{2, 5, 1, 4, 1, 1, 2, 8, 8\}$ bi lahko predstavili z:

\[
  \begin{tikzpicture}[level distance=0.9cm,
    level 1/.style={sibling distance=4cm},
    level 2/.style={sibling distance=2cm},
    level 3/.style={sibling distance=2cm}
    ]
    \node {\textbf{2} : 1}
      child {node {\textbf{1} : 3}}
      child {node {\textbf{5} : 1}
        child {node {\textbf{4} : 1}}
        child {node {\textbf{8} : 2}}
      };
  \end{tikzpicture}
\]

\podnaloga
Napišite tip \verb|'a mm_drevo|, ki vsebuje prazno drevo, in pa drevesa zgrajena iz vozlišč, ki vsebujejo element tipa \verb|'a|, števec tipa \verb|int| in pa levo in desno poddrevo.

\podnaloga
Napišite funkcijo \verb|vstavi : 'a mm_drevo -> 'a -> 'a mm_drevo|, ki dani element vstavi v multi-množico. Pri tem naj doda novo vozlišče, če elementa še ni v drevesu, sicer pa poveča primeren števec. Drevo naj ohrani strukturo dvojiškega iskalnega drevesa.

\podnaloga
Napišite funkcijo \verb|multimnozica_iz_seznama : 'a list -> 'a mm_drevo|, ki sestavi drevo, ki predstavlja multi-množico, ki vsebuje elemente danega seznama.

\noindent\textbf{Namig:} Funkcijo lahko uporabite, da ustvarite testne primere. 

\podnaloga
Napišite funkcijo \verb|velikost_multimnozice : 'a mm_drevo -> int|, ki vrne velikost multi-množice, ki jo drevo predstavlja.

\podnaloga
Napišite funkcijo \verb|seznam_iz_multimnozice : 'a mm_drevo -> 'a list|, ki vrne seznam, ki vsebuje vse elemente multi-množice. Za vse točke naj bo vrnjen seznam urejen, pri tem pa ni dovoljeno uporabiti urejevalnih funkcij.

\noindent\textbf{DODATNA NALOGA:} Za dosego dodatnih točk naj bo funkcija v celoti repno rekurzivna (ne pozabite preveriti katere vgrajene funkcije na seznamih so repno rekurzivne).

\prostor

%%%%%%%%%%%%%%%%%%%%%%%%%%%%%%%%%%%%%%%%%%%%%%%%%%%%%%%%%%%%%%%%%%%%%%%
\naloga[]
\emph{Nalogo lahko rešujete v Pythonu ali OCamlu.}

\vspace{5mm}
Žabica se je izgubila v močvari in želi kar se da hitro odskakljati ven. Na srečo močvara vsebuje veliko muh, s katerimi si lahko povrne energijo, kajti utrujena žabica ne skoči daleč.

Želimo ugotoviti, kako hitro lahko žabica odskaklja iz močvare. Močvaro predstavimo z tabelo, kjer žabica prične na ničtem polju. Če je močvara dolžine $k$, je cilj žabice priskakljati vsaj na $k$-to polje ali dlje (torej prvo polje, ki ni več vsebovano v tabeli).

Energičnost žabice predstavimo z dolžino najdaljšega možnega skoka. Torej lahko žabica z količino energije $e$ skoči naprej za katerokoli razdaljo med $1$ in $e$, in če skoči naprej za $k$ mest ima sedaj zgolj $e - k$ energije. Na vsakem polju močvare prav tako označimo, koliko energije si žabica povrne, ko pristane na polju. Tako se včasih žabici splača skočiti manj daleč, da pristane na polju z več muhami. Predpostavimo, da ima vsako polje vrednost vsaj $1$, da lahko žabica v vsakem primeru skoči naprej.

V primeru \verb|[2, 4, 1, 2, 1, 3, 1, 1, 5]| lahko žabica odskaklja iz močvare v treh skokih, v močvari \verb|[4, 1, 8, 2, 11, 1, 1, 1, 1, 1]| pa potrebuje zgolj dva. 

Za vse točke naj bo funkcija primerno memoizirana. Funkcija mora v nekaj sekundah izračunati rešitev za močvirje dolžine 50, ki ima na vsakem polju število 10.

\end{document}
