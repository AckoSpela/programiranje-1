\documentclass[arhiv]{../izpit}
\usepackage{fouriernc}
\usepackage{xcolor}
\usepackage{tikz}

\begin{document}

\izpit{Programiranje I: 1.\ izpit}{26.\ januar 2017}{
  Čas reševanja je 150 minut.
  \textbf{Funkcij v Haskellu ne pozabite opremiti z ustrezno signaturo.}
  Veliko uspeha!
}

%%%%%%%%%%%%%%%%%%%%%%%%%%%%%%%%%%%%%%%%%%%%%%%%%%%%%%%%%%%%%%%%%%%%%%
\naloga[Aritmetični izrazi, 30 točk]

%%%%%%%%%%%%%%%%%%%%%%%%%%%%%%%%%%%%%%%%%%%%%%%%%%%%%%%%%%%%%%%%%%%%%%
\naloga[Orbite, 20 točk]

\podnaloga
  Sestavite funkcijo \texttt{orbita}, ki izračuna seznam vseh različnih
  elementov, ki jih lahko dobimo z zaporedno uporabo dane funkcije na danem
  elementu. Na primer:
  \begin{verbatim}
  ghci> orbita (\x -> mod (x + 2) 10) 13
  [13,5,7,9,1,3]
  ghci> orbita succ 5
  [5,6,7,8,9,10,11,...]
  ghci> orbita negate 0
  [0]
  ghci> orbita negate 1
  [1,-1]
  \end{verbatim}

\podnaloga
  Sestavite funkcijo \texttt{generatorji}, ki izračuna najkrajši seznam vseh
  elementov, ki jih potrebujemo, da z zaporedno uporabo dane funkcije dobimo
  vse elemente danega seznama. Na primer:
  \begin{verbatim}
  ghci> generatorji negate [1,-2,-1]
  [1,-2]
  ghci> generatorji (\x -> (x + 3) `mod` 10) [1,2,3,4]
  [1]
  ghci> generatorji (\x -> (x + 2) `mod` 10) [1,2,3,4]
  [1,2]
  ghci> generatorji (\x -> (x + 2) `mod` 10) [1,2,3,4,5,11]
  [2,11] 
  \end{verbatim}
  Če je najkrajših seznamov več, lahko funkcija vrne katerega koli.


%%%%%%%%%%%%%%%%%%%%%%%%%%%%%%%%%%%%%%%%%%%%%%%%%%%%%%%%%%%%%%%%%%%%%%
\naloga[Fiksne točke, 10 točk]

%%%%%%%%%%%%%%%%%%%%%%%%%%%%%%%%%%%%%%%%%%%%%%%%%%%%%%%%%%%%%%%%%%%%%%
\naloga[Zobniki, 20 točk]
 
\end{document}

